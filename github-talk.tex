\documentclass{beamer}

\usetheme{Boulder}


\title{git and GitHub}
\subtitle{Using GitHub at MLSO}
\author{Michael D. Galloy}
\institute[NCAR/MLSO]{NCAR/HAO --- MLSO}

\date{29 July 2015}


\begin{document}


\begin{frame}{xkcd}
  \includegraphics[width=4in]{git_commit.png}
\end{frame}

\begin{frame}[plain]
  \titlepage
\end{frame}


\section{Introduction}
\subsection*{}

\begin{frame}
  \tableofcontents
\end{frame}

\begin{frame}{Subversion at NCAR}
CISL intends to stop hosting Subversion by Oct 1, 2016.

\vspace{1em}

Choices:
  \begin{enumerate}
    \item GitHub
    \item internally hosted git
  \end{enumerate}
For me, the reason to switch to git was GitHub, but git is much more powerful (and esoteric) than Subversion.
\end{frame}


\section{git}

\subsection{Distributed revision control}
\begin{frame}{git is distributed}
  \begin{enumerate}
    \item no central repo (but we can act like it)
    \item each repo is self-contained (though might not have everything)
    \item can do work locally and choose what to share to others
  \end{enumerate}
  \begin{center}
    \includegraphics[width=3in]{drc.pdf}
  \end{center}
\end{frame}

\begin{frame}{Terminology/concepts}
  \begin{description}
    \item[stage] put local files in the staging area, ready for the next commit
    \item[commit] make a local commit
    \item[hash] revision ``number'' of a commit
    \item[remote] another git repo, such as {\em origin}
    \item[push/fetch/pull] transfer commits, tags, etc. between local repo and a remote
  \end{description}
\end{frame}

\subsection{Normal workflow}
\begin{frame}[fragile]
  \frametitle{Cloning a repo}
To ``checkout'' an existing repo:
  \begin{small}
  \begin{lstlisting}
git clone https://github.com/ncar-mlso/comp-utilities.git
git clone git@github.com:mgalloy/test.git
  \end{lstlisting}
  \end{small}
The remote you clone from is typically called the {\em origin}.

\vspace{1em}

Upload your public key to GitHub --- keys make for less password typing. 
\end{frame}

\begin{frame}[fragile]
  \frametitle{Stage}
After making a local change, stage the files you want to commit:
  \begin{lstlisting}
git add <filename>
  \end{lstlisting}
Check to see status:
  \begin{lstlisting}
git status
  \end{lstlisting}
Git status messages show useful commands (such as how to stage a file, un-stage a file, or discard changes to a file)!
\end{frame}

\begin{frame}[fragile]
  \frametitle{Commit and push}
After staging, you are ready to commit:
  \begin{lstlisting}
git commit
  \end{lstlisting}
And push to GitHub:
  \begin{lstlisting}
git push origin master
  \end{lstlisting}
Here, {\em origin} is the remote to push to and {\em master} is the branch to push to.
\end{frame}

\begin{frame}[fragile]
  \frametitle{Pull}
Get changes from another repo:
  \begin{lstlisting}
git pull origin master
  \end{lstlisting}
Technically, this is a fetch and a merge, but is convenient in the normal case.

\vspace{1em}

Do this before starting work.
\end{frame}

\begin{frame}[fragile]
  \frametitle{Examine old commits}
  \begin{small}
  \begin{lstlisting}
$ git log --oneline
1dc418e Typo.
3512d91 Breaking COMP_PARSE_TIME out of COMP_EXTRACT_TIME.
ee7db9f Formatting.
798f6d5 Formatting.
01456d9 Basic top-level run.
5176a3a Updating library routines.

$ git checkout 5176a3a
  \end{lstlisting}
  \end{small}
\end{frame}


\subsection{Reverting changes}
\begin{frame}[fragile]
  \frametitle{Undo a local change}
Easy to find local changes:
  \begin{lstlisting}
git diff
git diff <filename>
  \end{lstlisting}
Use {\tt --staged} to compare differences in staged files.

\vspace{1em}

If you have made a local change that you want to revert, but haven't committed yet:
  \begin{lstlisting}
git checkout -- <filename>
  \end{lstlisting}
\end{frame}

\begin{frame}[fragile]
  \frametitle{New commit that undoes existing commits}
For example, to undo the last two commits:
  \begin{lstlisting}
git revert HEAD~2..HEAD
  \end{lstlisting}
Or to revert back to commit 0123456:
  \begin{lstlisting}
git revert 0123456..HEAD
  \end{lstlisting}
These commands create a new ``inverse'' commit.
\end{frame}

\begin{frame}[fragile]
  \frametitle{Revert existing commits}
Undo last commit (but keep changes):
  \begin{lstlisting}
git reset --soft HEAD~1
  \end{lstlisting}
Completely remove last commit and changes:
  \begin{lstlisting}
git reset --hard HEAD~1
  \end{lstlisting}
These are problematic if the commits have been pushed to a remote.
\end{frame}

\begin{frame}{Other things you can do}
  \begin{enumerate}
    \item make a branch
    \item merge branches
    \item make a tag
    \item stash temporarily
    \item bisect
    \item and many more\ldots
  \end{enumerate}
\end{frame}

\subsection{Resources}
\begin{frame}{Resources}
  \begin{itemize}
    \item Just starting:
      \begin{itemize}
        \item https://try.github.io/
        \item https://www.atlassian.com/git/tutorials/
      \end{itemize}
    \item Full reference:
      \begin{itemize}
        \item Pro Git book: http://git-scm.com/book
      \end{itemize}
  \end{itemize}
\end{frame}

\section{GitHub}
\subsection*{}

\begin{frame}{Features}
GitHub is a social code hosting site. It features:
  \begin{itemize}
    \item browsing files, commits, branches, tags, etc.
    \item issues tracker
    \item wiki pages (actually another git repo)
    \item social: watch/star repos, follow people
  \end{itemize}
\end{frame}

\begin{frame}{OS clients}
The GitHub client applications can handle non-GitHub repos as well.

\vspace{1em}

Get the appropriate client here:
  \begin{itemize}
    \item https://mac.github.com
    \item https://windows.github.com
  \end{itemize}
Sorry Linux! (The command line is better anyway.)
\end{frame}

\begin{frame}{OS clients}
  \includegraphics[width=4in]{mac-client.png}
\end{frame}


\section{Conclusion}
\subsection*{}

\begin{frame}{Conclusion}
  \begin{enumerate}
    \item While git is more powerful than Subversion, you can use it like Subversion
    \item GitHub provides an easy browser with social integration and other code management features such as an issue tracker and wiki
  \end{enumerate}
\end{frame}

\begin{frame}{Thanks!}
  \begin{center}{\huge Thanks!}\end{center}
  \begin{itemize}
    \item {\tt mgalloy@ucar.edu}
  \end{itemize}
\end{frame}



\end{document}
